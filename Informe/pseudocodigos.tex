\subsection{Eliminación Gaussiana de la matriz banda}

{\tiny

\lstset{caption=Pseudocódigo Eliminación Gaussiana con matriz banda, label= Pseudocódigo Eliminación Gaussiana con matriz banda, framexleftmargin=2mm, frame=shadowbox, rulesepcolor=\color{black}, language={[ANSI]C}}
\lstinputlisting[language=C++]{pseudocodigo/banda.cpp}

}

~

{\normalsize
Aclaraciones respecto al método de Eliminación Gaussiana:

Las funciones ''cargar\_n\_igual'' son las que inicializan la matriz con los valores correctos.

En resolver sistema mantenemos un vector de diagonales, donde $diagonales_{i}$
guarda la posición de la diagonal del elemento $i$ (inicialmente 3). Cuando se realiza el swapeo de filas, las diagonales
cambian.
}

\subsection{Heurística}

Antes de empezar, hagamos un repaso de las estructuras auxiliares para resolver la heurística:

{\tiny

\lstset{caption= Estructuras de la heurística, label= Estructuras de la heurística, framexleftmargin=2mm, frame=shadowbox, rulesepcolor=\color{black}, language={[ANSI]C}}
\lstinputlisting[language=C++]{pseudocodigo/estructuras.h}

}

{\tiny

\lstset{caption=Pseudocódigo de la heurística , label= Pseudocódigo de la heurística, framexleftmargin=2mm, frame=shadowbox, rulesepcolor=\color{black}, language={[ANSI]C}}
\lstinputlisting[language=C++]{pseudocodigo/heuristica.cpp}

}

~

Aclaraciones respecto de la heurística:

La salida del método es un pair< pair<double costoTotal,bool esSeguro>, vector<int> posicionesPilares>, donde
costoTotal es el costo total del puente: costo de los cp\_e + $k$*costo\_pilar, siendo $k$ la cantidad de pilares insertados.

esSeguro indica si el puente formado por todas las sub-estructuras es seguro o no. En el caso de no serlo no significa
que la heurística sea mala (ya que en el peor caso, nuestro algoritmo genera muchos puentes de 2 secciones), sino
que indica que a pesar de todo la inserción de pilares no pudo contrarrestar el peso de las cargas. Recordemos
el caso de un puente de 2 secciones con su única carga muy pesada (la heurística no puede hacer nada al respecto).

posicionesPilares contiene las posiciones ''relativas'' de los pilares insertados. Es decir, si por ejemplo divido el puente
en 2, y luego divido la mitad derecha en 2, la pisición absoluta del tercer pilar es $\frac{3}{4}$. Sin embargo, en el 
vector queda guardado $\frac{1}{2}$, ya que se insertó en la mitad del puente derecho. Sin embargo, lo que importa es que
el tamaño de este vector determina la cantidad de pilares insertados.

Finalmente: calcular\_indice\_insercion se comporta acorde a como había sido presentado. En el caso de que el peso de las
cargas sea uniforme intenta ubicar el pilar en el medio de la estructura (cuidándose de que queden 2 sub-estructuras
válidas). En caso contrario, intenta ubicar un pilar que reemplace la carga más pesada.