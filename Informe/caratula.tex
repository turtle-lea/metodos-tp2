Integrantes:
\begin{itemize}
	\item Castro, Dami\'an L.U.: 326/11  \verb+ltdicai@gmail.com+
	\item Matayoshi, Leandro L.U.: 79/11 \verb+leandro.matayoshi@gmail.com+
	\item Szyrej, Alexander L.U.: 642/11   \verb+alexanderszyrej@gmail.com+
	
\end{itemize}

\vspace{0.5cm}

\begin{abstract} 
	El siguiente trabajo práctico tiene como objetivo analizar el comportamiento de la representación de una estructura 
llamada
puente \emph{''Pratt Truss''} al realizar experimentos en donde variamos distintos factores que pueden modificar su
comportamiento, como por ejemplo el peso ejercido sobre el puente, el largo del mismo, la altura, etc. Para trasladar
nuestro problema hacia el campo de los métodos numéricos generamos un modelo matemático de
representación y establecemos una equivalencia entre encontrar la solución de nuestro problema y resolver el sistema
$Ax = b$, en donde $A \in R^{nxn}$, $x \in R^{n}$ y $b \in R^{n}$.

	El método elegido para la resolución del sistema es ''Eliminación Gaussiana'' con pivoteo parcial.
Demostramos que debido a las condiciones particulares del modelo y del problema,
	 la matriz resultante es banda $p,(q+p)$ por lo
que aprovechamos este aspecto para implementar un algoritmo más eficiente en cuanto a tiempo y memoria. 

	Finalmente implementamos un método heurístico para los casos en donde nuestro análisis detecta una estructura
insegura (esto sucede cuando una fuerza superior a la máxima soportada es ejercida sobre uno de los links del puente).
El algoritmo se basa en la inserción de pilares para generar sub-estructuras más estables, aunque intentando minimizar
la cantidad de inserciones ya que es un procedimiento costoso.
\end{abstract}

\vspace{0.5cm}

Palabras Clave:
\begin{itemize}
	\item Puente ''Prat Truss''
	\item Eliminación Gaussiana con pivoteo parcial	
	\item Matriz banda $p,q$
	\item Heurística
\end{itemize}

