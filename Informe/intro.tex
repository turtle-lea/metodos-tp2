	En este trabajo práctico se nos pidió analizar el comportamiento de un modelo de una estructura conocida como 
puente Pratt Truss al simular la aplicación de peso sobre el mismo.
	El modelo consiste en una representación en 2 dimensiones del puente. Llamamos \emph{span} al largo de la base 
y \emph{h} a la altura del mismo. A su vez cuenta con una cantidad par \emph{n} de secciones, $2*n$
juntas, $4*n-3$ links y 3 fuerzas adicionales que actúan sobre el puente, a las que llamamos $h_{0}$, $v_{0}$ y
$v_{1}$.
	A su vez, en las $(n-1)$ juntas inferiores centrales es posible insertar cargas que representan el peso 
ejercido sobre dichas juntas.
	Este modelo nos permite analizar la evolución del sistema al alterar cada uno de los distintos factores,
como el \emph{span}, la altura y el peso de las cargas.
	Concretamente el objetivo fue diseñar un algoritmo que, dada una instancia de puente Pratt Truss con una configuración de
cargas iniciales calcule el valor de las fuerzas ejercidas sobre cada uno de los links.
	Para ello planteamos un sistema de ecuaciones (la deducción del mismo aparece explicada en el desarrollo) y
establecemos una equivalencia entre resolver el problema y hallar la solución de $Ax = b$.

1) $A$ representa la forma en que interactúan los links sobre cada una de las juntas. Cada una de las filas de la matriz 
representa una ecuación de fuerzas que actúan sobre la junta (una para el plano vertical y otra para el horizontal). Sabemos
que las fuerzas actuantes en cada plano para cada junta deben estar en equilibrio.

2) \emph{x} es el vector de $4n$ fuerzas que constituyen las incógnitas de nuestro problema.

3) \emph{b} es un vector de $4n$ valores. Estas fuerzas intervienen en las ecuaciones mencionadas en el punto (1). Las fuerzas
actuantes en cada plano para cada junta deben estar en equilibrio (Es decir, la suma de las mismas debe ser igual a 0). Las
cargas forman parte de este vector (actúan sobre el plano vertical). Los valores de \emph{b} para las ecuaciones
correspondientes al plano horizontal son iguales a 0.

Es posible demostrar que la matriz \emph{A} cumple la propiedad de ser banda $p,(q+p)$. Por lo tanto utilizamos en nuestros
algoritmos una representación que nos permite ganar eficiencia en cuestiones de tiempo y espacio.
El método seleccionado para resolver el problema fue Eliminación Gaussiana con pivoteo parcial. Una buena explicación del
mismo puede ser encontrada en el libro ''Numerical analysis'',\emph{Burden, Faires}. El método implementado está adaptado
a la representación de la matriz elegida.

	La última parte del trabajo consistió en desarrollar un método heurístico para aplicar en los casos donde el 
máximo de los módulos de las fuerzas halladas supere cierto valor (parámetro de entrada). La idea es
modificar la estructura insertando pilares (y de esta manera generando subestructuras)
para tratar de reducir la magnitud de las fuerzas ejercidas sobre los links, aunque evitando insertar pilares innecesarios
ya que los mismos son costosos. Una heurística es un algoritmo que, si bien no necesariamente encuentra la solución óptima
a nuestro problema (por ejemplo, porque esto podría ser muy costoso) propone una solución aceptable.
