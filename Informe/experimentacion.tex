\subsection{Experimentación}

Luego de implementar el algoritmo de eliminación Gaussiana con pivoteo parcial adaptado a nuestra representación de la matriz banda,
realizamos experimentos para ver qué sucedía con las fuerzas ejercidas sobre cada link en base a lo pedido por el enunciado:

a) Variando el \emph{span}, con la altura y el peso de las cargas fijo, para distintos valores de \emph{n}.

b) Variando el peso de las cargas, con el \emph{span} y la altura fijos, para distintos valores de \emph{n}.

Realizamos ambas etapas de la experimentación variando \emph{n} (que representa la cantidad de secciones de nuestro puente).
Las pruebas fueron realizadas para valores de $n= 4,6,8,10,12$ (valores relativamente peque\~nos, con granularidad de 2), pero 
también con $n=16,32,64,128$ (granularidad mayor).

La parte b) de la experimentación se divide a su vez en otras 3 partes, ya que hay varias maneras de variar el peso de las cargas: 

1) Modificando solamente el peso de una de las ellas, dejando a las restantes con igual peso.

2) Variando uniformemente el peso de todas las cargas.

3) Variando únicamente el valor de la carga de la junta del medio de la estructura.

Decidimos separar la sección b) en estas 3 partes para poder corroborar algunas de nuestras hipótesis o simplemente observar 
los resultados:

1) Hipótesis: Al aumentar el peso de una carga aplicada sobre una $junta_{i}$, existe algún/algunos links en la estructura que se ven
particularmente afectados.

2) Observación: Determinar cómo aumentan las fuerzas ejercidas sobre los links a medida que aumenta
el peso de las cargas aguantado por la estructura.

3) Observación: Comparar el crecimiento de las fuerzas ejercidas sobre los links cuando aumenta el peso sobre todas las juntas respecto al
crecimiento cuando aumenta el peso únicamente sobre la junta del medio. Además:
Hipótesis: Variar únicamente el peso de la carga sobre la junta del medio debería mantener la simetría de las fuerzas aplicadas sobre
los links de toda la estructura.

Finalmente, la hipótesis inicial planteada para la parte a) consistió en que, a medida que se incrementa el \emph{span} las fuerzas
ejercidas aumentan también.

(Aclaración: cuando decimos ''las fuerzas aumentan'' nos referimos al módulo de las mismas)

