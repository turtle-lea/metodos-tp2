	Los experimentos realizados demuestran que las magnitudes de las fuerzas aplicadas sobre los links de la estructura
varían en función del largo de la estructura, de la cantidad de secciones, del peso de las cargas y de la distribución
de las mismas (No es lo mismo concentrar gran parte del peso en el medio de la estructura, que sobre uno de los extremos,
o de manera uniforme). En particular las magnitudes aumentan para estructuras con un \emph{span} más largo y, lógicamente,
cuando se aplica más peso sobre la misma.

	La heurística constituye una buena solución a nuestro problema,
ya que se basa en los experimentos mencionados anteriormente. Intenta eliminar las cargas más pesadas (mayores en módulo)
reemplazándolas por pilares, y en el caso de que el peso esté uniformemente distribuído ubica un pilar en el medio,
reduciendo el \emph{span} de ambas subestructuras. Sin embargo hay casos en donde no puede generar un puente seguro,
pero ningún otro algoritmo podría. Por ejemplo, un puente de tan solo 2 secciones en donde la única carga aplicada
es demasiado pesada. (2 es la mínima cantidad de secciones de un puente ''Prat Truss'').

	Aplicar ''Eliminación Gaussiana'' para resolver un problema concreto de la vida real resultó motivante y de gran
interés. En suma, encontrar una buena representación de la matriz banda y luego adaptar el método fue un desafío
bastante grande.

	Finalmente podemos mencionar que la primera parte del trabajo, en donde planteamos el sistema e inicializamos
la matriz fue sin dudas la más difícil, ya que es un procedimiento que hicimos de forma casi ''manual'', y por lo
tanto encontrar
errores en esta parte del código resultó ser una tarea ardua y trabajosa.